\documentclass{strrespaper-journ}
\usepackage[utf8]{inputenc}
\usepackage[T1]{fontenc}
\usepackage{csquotes}
\usepackage[english]{babel}

% For better tables
\usepackage{booktabs}
\usepackage[flushleft]{threeparttable}

\newcommand{\mnk}{\textit{m, n, k} game}
\newcommand{\mnkpl}{\textit{m, n, k} games}
\newcommand{\ttt}{Tic-Tac-Toe}
\newcommand{\TTT}{TIC-TAC-TOE}

% For images
\usepackage{graphicx}

% For plotting
\usepackage{xcolor}
\usepackage{pgfplots}
\pgfplotsset{compat=1.16}
\usepackage{tikz}

% For scientific names/acronyms (type "texdoc glossaries" [without the quotes] for more information)
\newscientificname{sname}{S. name}{Scientific name}

% For units
\usepackage{siunitx}
\sisetup{
	separate-uncertainty=true
}

% TODO: Feel free to remove once your remove all the dummy text
\usepackage{blindtext} 

% Required for the citations
\usepackage[style=apa,sortcites=true,sorting=nyt,backend=biber]{biblatex}
\addbibresource{../str-templates/sample-resources/bibliographies/str.bib} % TODO: Add your BibLaTeX files here

\title{Comparison of Genetic Algorithm Training Methods as Applied to Tic-Tac-Toe} % TODO: Title here
\runningHead{Comparison of Genetic Algorithm Training Methods} % TODO: 50 characters max for running head

\addAuthor{Vash Patrick B. Ancheta} % TODO: Student 1
\addAuthor{Diego Sulayman R. Pascua} % TODO: Student 2
\addAuthor{Resh Vnzi S. Togue\~no} % TODO: Student 3

\addAuthor{Kaye Melina Natividad B. Alamag} % TODO: Research Adviser
\addAuthor{Jay Jay F. Manuel} % TODO: Research Teacher

\affiliation{
	Philippine Science High School -- Cordillera Administrative Region Campus, Purok 12, Irisan,\\
	Baguio City, 2600, Philippines
}  % TODO: Affiliation

\email{vashpatrickancheta@gmail.com} % TODO: Correponding author email

\abstract{
	Machine learning methods are algorithms where machines are not explicitly programmed to do what is tasked but rather, learns how to perform the task.
			An \mnk\ is a game where there is an $m \times n$ grid and two players alternate turns trying to earn $k$ pieces adjacent to each other horizontally, vertically or diagonally.
			The \mnk\ to be used in the research to test the MLMs is \ttt, configured as 3,3,3.
			The study utilizes an existing genetic algorithm to be used as a control setup.
			This genetic algorithm is modified to be controlled by move generators using the Controlled Elite Preservation operator and the resulting genetic algorithms are compared with regard to performance.
			Using an ANOVA Test at $\alpha = 0.05$, no significant difference in performance was found between the unmodified and modified genetic algorithms.
			This study provides a backbone for research involved in the transmission of knowledge between \enquote{smart} artificial intelligence and \enquote{naive} intelligence, raising the question on whether the evolution of a genetic algorithm can be better affected by a move generator in other components.
} % TODO: Abstract
\keywords{genetic algorithm; machine learning; tic-tac-toe; training methods} % TODO: Should be arranged alphabetically and separated by semicolon (This example is not)

\begin{document}
	\maketitle

	\section{Introduction}
		Machine learning (ML) is vast---it is used in different situations such as spam detectors, web search engines, photo tagging applications and game development \autocite{sharmaMachineLearningApplications2016}.
			There have been researches that are aimed at improving the implementation of ML in various games.
			A category of games under investigation through ML is the set of \mnk\ games, comprised of games where there is an $m \times n$ grid and two players alternate turns trying to earn $k$ pieces adjacent to each other horizontally, vertically or diagonally \autocite{hayesDevelopingMemoryEfficient2016}.
			Among the most common examples of \mnkpl\ are G\=o, Othello, and Chess.
			\ttt, the game under investigation in this study, is an example of an \mnk.
			A \ttt\ board is composed of three rows and three columns, and requires three adjacent pieces of the same player to render a win, thus it is considered to have a $3, 3, 3$ configuration.

			Improvements in ML have lead to the development of artificial intelligence (AI) players that can beat even the most competitive human players around the world.
			Machine learning methods (MLMs) are algorithms where machines are not explicitly programmed to do what is tasked.
			Rather, similar to its namesake, MLM-trained machines are capable of performing tasks given its own internal code without any human interference.
			In short, the machine \textit{learns} \autocite{geeksforgeeksMachineLearning}.
			An example of an MLM is the genetic algorithm (GA).

			This study aims to develop multiple GAs with different elite preservation methods and compare their performance in \ttt\ based on the possible situations.

		 To compare the effectiveness of trained genetic algorithm (GA) organisms among each other as applied to \ttt
				\end{itemize}
			\subsection{Specific Objectives}
				\begin{enumerate}
					\item To implement known heuristics into Python code
					\item To train organisms of an implemented GA using different move generators (MGs)
					\item To compare the development of the performance of trained GA organisms among each other within 500 generations
				\end{enumerate}

		
			
			This study contributes to the body of knowledge in ML.
			Through this study, more can be known about how information gained from one method of AI can be passed on to another mechanism of AI through training.
			This sheds light on how information from one AI player can be transmitted to an MLM such as GA.
			This, by extension, can improve the comprehension of how machines can learn strategies in games from one with greater skill.

		
			
			This study focuses only on \ttt\ and not other games such as Chess or G\=o because it is the simplest game to conduct the research on heuristics, namely the training of the GA under different move generators (MGs).
			The complexity of the board game is not relevant to the study because the focus of the research is to compare the effectiveness of trained GA organisms given an \mnk.
			Applying these heuristics on other \mnkpl\ however are beyond the time frame of the research.
			Only three GAs were developed in this study.
			The first is a Python implementation of the GA in the work of \textcite{bhattSearchNolossStrategies2008}.
			Using developed AI, the second and third are modified implementations of the same GA.
			The performance of each GA is based on how many generations it takes for the GA to find a no-loss first player for \ttt.
			This basis for comparison of performance, specifically using the skill of an organism as a first player, is due to the fact that Python is known for being slow.
			In line with this, indices are 0-based in this paper, as they are in Python.

	\section{Methodology}
		% TODO: Methodology
		\subsection{Subheading for Procedure 1}
			Methods should be described concisely and clearly to allow experiments to be repeated.
			For commonly used methods, a simple reference is sufficient.
			Avoid references that are not readily accessible.

	\section{Results}
		% TODO: Results
		\subsection{Subheading for Result 1}
			In Results, present data in only one of the following: text, table, or figure.
			Results should preferably have no more than five illustrations (tables and/or figures).
			Do not use tables or figures to present data that can be more concisely stated in the text.
			Discussion should interpret results in relation to previously published work.
			Do not repeat results or reiterate the introduction.
			\begin{table}[htbp]
				\centering
				\begin{threeparttable}
					\caption{Table Label Should be Concise and in Title Case \tnote{a}}
					\label{tab:concise_table}
					\begin{tabularx}{\linewidth}{cXXX}
						\toprule
						Criteria Number & A (\si{\celsius}) & B (\si{\kilo\meter}) & C (\si{\ampere}) \\
						\midrule
						1               & \num{1.00(1)}     & \num{1.00(1)}        & \num{1.00(1)}    \\
						2               & \num{1.00(1)}     & \num{1.00(1)}        & \num{1.00(1)}    \\
						3               & \num{1.00(1)}     & \num{1.00(1)}        & \num{1.00(1)}    \\
						4               & \num{1.00(1)}     & \num{1.00(1)}        & \num{1.00(1)}    \\
						\bottomrule
					\end{tabularx}
					\begin{tablenotes}
						\small
						\item[a] Note: Type each table on a separate sheet.
						Never use vertical lines to separate columns.
						Prepare tables so that compared data read down, not across.
						Columns that show no significant variations should be omitted.
						Do not use tables for word lists.
						Titles should be clear, and column headings should be brief with units of measurements in parentheses.
						Symbols and abbreviations should be defined below the table.
						Indicate table footnotes with a, b, c, etc.
						Do not present the same data in both graphical and tabular form.
						Tables should be self-explanatory or understandable without reference to the text.
					\end{tablenotes}
				\end{threeparttable}
			\end{table}

			\begin{figure}[htbp]
				\centering
				\definecolor{grrayt}{HTML}{848484}
				\begin{tikzpicture}
					\begin{axis}[
							width  = \linewidth,
							height = 8cm,
							major x tick style = transparent,
							ybar=2*\pgflinewidth,
							bar width=\linewidth/12,
							ymajorgrids = true,
							ylabel = {Softening point (\si{\celsius})},
							symbolic x coords={0\% PP, 2\% PP, 4\% PP, 6\% PP},
							xtick = data,
							scaled y ticks = false,
							try min ticks = 10,
							enlarge x limits=0.25,
							ymin=0, ymax = 100
						]
						\addplot[style={grrayt,fill=grrayt,mark=none}, error bars/.cd,
							y dir=both, y explicit, error bar style={color=black}]
						coordinates {(0\% PP, 48.5)+-(4.0, 4.0) (2\% PP, 56.0)+-(2.0, 2.0) (4\% PP, 61.5)+-(2.5, 2.5) (6\% PP, 68.0)+-(2.0, 2.0)}
						node[pos=0/4,anchor=south, color=black, yshift=10] {48.5}
						node[pos=1/4,anchor=south, color=black, yshift=10] {56.0}
						node[pos=3/4,anchor=south, color=black, yshift=10] {61.5}
						node[pos=4/4,anchor=south, color=black, yshift=10] {68.0};
					\end{axis}
				\end{tikzpicture}
				\caption{%
					Figure label should be in sentence case ending in period, placed under the figure.
					Each figure should have a legend or caption and should be self-explanatory.
					Figure legends should be in lowercase print-type, except for the first letter of first word.
					Abbreviations and symbols on figures should be defined in the legend.
					Figures include line drawings, photographs, and computer plots.
					(The graphics should be clear)%
				}
				\label{fig:bar_graph}
			\end{figure}

	\section{Discussion}
		% TODO: Discussion
		Discussion should incorporate referencing of the figures or data tables presented in the Results section.
		Literature citations should be selective, not to exceed 30 references for a research paper.

	\section{Summary and Conclusion}
		% TODO: Summary and Conclusion
		Point out significant findings of the study in relation to the objectives of the study.
		Include your recommendations in this section.

	\section{Acknowledgement}
		% TODO: Acknowledgement
		Acknowledgements should be brief and are placed under a separate heading immediately before References.
		Acknowledge any financial support for the work being published and personal assistance.

	\section{Demonstration}
		% TODO: Remove this section
		\subsection{This is a demo}
			This is a demonstration of the depth of headings one might use.
			\subsubsection{Paragraph}
				\paragraph{I am a paragraph}
					\blindtext
					\subparagraph{I am a subparagraph}
						\blindtext

		\subsection{Another subsection}
			\blindtext[2] \autocite{letcherWindEnergyEngineering2017}
			\nocite{al-shemmeriWindTurbines2010}
			\nocite{trewbyWindEnergyImplications2014}

	\printbibliography
		% The example bibliographical entries were in APA 6th so it is recommended that you simply generate a .bib file using software such as Zotero and use BibLaTeX to format the citations and bibliography for you.
		% It was noted in the original template that Personal communication should appear only in text parenthetically with the names of persons who supplied the information. They should not be listed in the References section.
\end{document}
